%%% LaTeX Template: Newsletter
%%%
%%% Source: http://www.howtotex.com/
%%% Feel free to distribute this template, but please keep the referal to HowToTeX.com.
%%% Date: September 2011


%%% ---------------
%%% PREAMBLE
%%% ---------------
\documentclass[10pt,a4paper]{article}

% Define geometry (without using the geometry package)
\setlength\topmargin{-48pt}
\setlength\headheight{0pt}
\setlength\headsep{25pt}
\setlength\marginparwidth{-20pt}
\setlength\textwidth{7.0in}
\setlength\textheight{9.5in}
\setlength\oddsidemargin{-30pt}
\setlength\evensidemargin{-30pt}

\frenchspacing						% better looking spacing

% Call packages we'll need
\usepackage[english]{babel}			% english
\usepackage{graphicx}				% images
\usepackage{amssymb,amsmath}		% math
\usepackage{multicol,caption}				% three-column layout
\usepackage{url}					% clickable links
\usepackage{marvosym}				% symbols
\usepackage{wrapfig}				% wrapping text around figures
\usepackage[T1]{fontenc}			% font encoding
\usepackage{charter} 				% Charter font for main content
\usepackage{blindtext}				% dummy text
\usepackage{datetime}				% custom date
	\newdateformat{mydate}{\monthname[\THEMONTH] \THEYEAR}
\usepackage[colorlinks=false]{hyperref}	% links and pdf behaviour
\usepackage{hyperref}



% Customize (header and) footer
\usepackage{fancyhdr}
\pagestyle{fancy}
\fancyhf{}
\lfoot{}
%\lfoot{	\footnotesize 
%		Newletter from HowToTeX.com \\
%		\Mundus\ \href{http://www.howtotex.com}{HowToTeX.com}	\quad
%		\Telefon\ 555-5555											\quad
%		\Letter\ \href{mailto:frits@howtotex.com}{frits@howtotex.com}
%	  }
\cfoot{}
\rfoot{\footnotesize ~\\ Page \thepage}
\renewcommand{\headrulewidth}{0.0pt}	% no bar on top of page
\renewcommand{\footrulewidth}{0.4pt}	% bar on bottom of page

%%% ---------------
%%% DEFINITIONS
%%% ---------------

% Define separators
\newcommand{\HorRule}[1]{\noindent\rule{\linewidth}{#1}} % Creating a horizontal rule
\newcommand{\SepRule}{\noindent							 % Creating a separator
						\begin{center}
							\rule{250pt}{1pt}
						\end{center}
						}						

% Define Title en News input
\newcommand{\JournalName}[1]{%
		\begin{center}	
			\Huge \usefont{T1}{augie}{m}{n}
			#1%
		\end{center}	
		\par \normalsize \normalfont}
		
\newcommand{\JournalIssue}[1]{%
		\hfill \textsc{\mydate \today, No #1}
		\par \normalsize \normalfont}

\newcommand{\NewsItem}[1]{%
		\usefont{T1}{augie}{m}{n} 	
		\large \bf #1 \vspace{4pt}
		\par \normalsize \normalfont}
		
\newcommand{\NewsAuthor}[1]{%
			\hfill by \textsc{#1} \vspace{4pt}
			\par \normalfont}		

\newcommand\sect[1]{%
  \section*{#1}%
  \addcontentsline{toc}{section}{#1}}

\newcommand\subsect[1]{%
  \subsection*{#1}%
  \addcontentsline{toc}{subsection}{#1}}


\newcommand{\HRule}{\rule{\linewidth}{0.5mm}}


\newenvironment{Figure}
  {\par\medskip\noindent\minipage{\linewidth}}
  {\endminipage\par\medskip}

%%% ---------------
%%% BEGIN DOCUMENT
%%% ---------------
\begin{document}



\begin{titlepage}

\begin{center}


% Upper part of the page
\includegraphics[width=0.5\textwidth]{pics/ecplogo.jpg}\\[1cm]    

\HRule \\[0.4cm]
{ \Huge \bfseries THE EXPLORER}\\[0.4cm]

\HRule \\[1.5cm]


\textsc{\LARGE July 2013}\\[9cm]

\textsc{\large The EXPLORER is the monthly newsletter of the Explorers Club Of Pittsburgh,Inc., a non-profit organization devoted to research, adventure and education in outdoor and wilderness recreation and conservation}\\[0.5cm]


% Title

% Author and supervisor
%\begin{minipage}{0.4\textwidth}
%\begin{flushleft} \large
%\emph{Author:}\\
%John \textsc{Smith}
%\end{flushleft}
%\end{minipage}
%\begin{minipage}{0.4\textwidth}
%\begin{flushright} \large
%\emph{Supervisor:} \\
%Dr.~Mark \textsc{Brown}
%\end{flushright}
%\end{minipage}

\vfill

% Bottom of the page
%{\large \today}

\end{center}

\end{titlepage}


% Title	
% -----
\JournalIssue{1}
\JournalName{The EXPLORER}
\noindent\HorRule{3pt} \\[-0.75\baselineskip]
\HorRule{1pt}
% -----

\tableofcontents

\clearpage


% Front article
% -----
\vspace{0.5cm}
	\SepRule
\vspace{0.5cm}



\begin{center}
\begin{minipage}[h]{0.8\linewidth}
	\begin{wrapfigure}{l}{0.41\textwidth}
		\includegraphics[width=0.6\linewidth]{pics/me.jpg}
		\\% this spacer is needed to make the text on the right fit OK
	\end{wrapfigure}
	
	\NewsItem{Message from the editor}

	\emph{Greetings, ECP!} Welcome to the June edition of the club newsletter. In this edition we have a trip report by Shane Shin about backpacking on the Blackforest Trail. Club business pertaining to the past general and BOG meetings are documented. Agenda for the upcoming June general meeting are presented. The Annual ECP audit committee report and the club treasury report are included as well.  Upcoming events of interest are advertised at the end of the magazine. Finally, contact information of the officers and activity coordinators is provided in the appendix. 
\\
\\
-- Subhodeep Moitra (Deep)

\vspace{0.5cm}



	Again as a reminder, you are invited to attend the club general meeting and the BOG meeting as well. The general meeting will have a cookout (basics provided by club), attendees are encouraged to bring side dishes, beer and cups. A scheduled slideshow/presentation will follow. Pre-meeting activities are encouraged such as trail/road run, road ride (mtn biking not good in Schenley), road ride, frisbee golf, etc. 
	
\vspace{1cm}

\begin{multicols}{2}
\Large



JUNE GENERAL MEETING\\
Thursday, July 11, 7:30PM\\
David Lawrence Shelter\\
Schenley Park, Pittsburgh
\\
JULY BOG MEETING\\
Thursday, July 18, 7:30PM\\
Rush Howe's Residence \\ 
Location to be announced

\normalsize
\end{multicols}
	
\end{minipage}
\end{center}
% -----

\pagebreak
\clearpage


% Other news (1)
% -----
\vspace{0.5cm}
	\SepRule
\vspace{0.5cm}
\begin{multicols}{2}


\sect{Trip Reports}

\subsect{Seneca Grad Trip - Megan Olson}

Hey everyone! Hope you had as amazing of a weekend as I did. Maybe I'm easy to please, but I enjoyed every second! As per my request, we didn't do anything technical but just had a relaxing time and in the process got a great tour of the rocks. On Saturday, my leader, Brian Dunlavey, my second, Brian Kent ("Brian to Brian"...), and I got to do the awesome Stairmaster (...), then went up Old Ladies.

We then did a little walking and a pitch or two to get up to the summit, which was even better than I could have expected. So peaceful (miraculously we were the only people up there at the time) - sun shining, light breeze... Heaven. After making my two Brians pose for photos I'm sure they didn't want to take (but for which they were good sports and smiled anyway), we all stood shoulder-to-shoulder on a teeny ledge and rapped down the back (east side) of the summit. Yes, I know everyone thinks I'm crazy because I like to rappel, but seriously - this was amazing! We then went up upper Broadway (I'm pretty sure) to the Gunsight notch in hopes of getting up the climb before the end of the day. On the way up we did one of the more "exposed" pieces of climbing I experienced all weekend (although it was just a few moves) - kind of out on the edge and face of a fin w/ nothing below but air. Pretty stunning.

\begin{Figure}
 \centering
 \includegraphics[width=0.95\linewidth]{pics/meg1.jpg}
 \captionof{figure}{At the base of the climb with Brian Dunlavey}
\end{Figure}

Unfortunately when we got to Gunsight there was one group on and another waiting, and since it was late in the afternoon we decided we needed to pack it in. We set up another rappel off a tree near the Gunsight notch, and I again enjoyed myself, yes.

Sunday we were up early again (I think I was the only one who hadn't drunk too much beer the night before? and decided to skip the Stairmaster (we didn't want to spoil ourselves w/ too much of a good thing) to do Skyline Traverse instead. This was an easy but fun climb w/ good variety, including the moment you stem out between the main rock and a pillar and are standing over thin air.

\begin{Figure}
 \centering
 \includegraphics[width=0.95\linewidth]{pics/meg2.jpg}
 \captionof{figure}{At the top with Brian Kent}
\end{Figure}


Similar to the day before, we worked our way around lower/upper Broadway, doing a pitch here, a pitch there (including a crack that we're pretty sure no one has touched in 50 years, maybe more?) to again get to Gunsight. This time there wasn't a group climbing, but we were past our decided turn-around time, so Gunsight will have to wait for another day.

\begin{Figure}
 \centering
 \includegraphics[width=0.95\linewidth]{pics/meg3.jpg}
 \captionof{figure}{Rappelling}
\end{Figure}


I really had no expectations for the weekend (mostly b/c I knew nothing about Seneca and decided to wait until I saw it in person to find out), but it was definitely cooler than I could have imagined. Brian \#1 was an awesome leader, patiently and calmly making sure we were always safe; and he taught me a lot - even simple stuff like how you logistically get three people up and around a mountain safely w/ your gear, rope management, history of climbing and of the Seneca area, etc. It's a great feeling to be totally comfortable 1,000 feet off the ground b/c you trust who you're w/ and you know they're knowledgable and are looking out for you. So great job, Brian to Brian!

Annnnnd the fact that I got to go swimming for the first time this summer in a crystal clear river and saw a really cute (albeit possibly poisonous) snake only made the weekend that much better. Plus, check out the weather we had... Unreal. What a weekend!
Thanks to everyone who made the trip possible (esp. Paul, Matt and my team) - I know organizing that many people to do something as "serious" as climbing is no small feat, but I'm so grateful you are willing to do it!

\subsect{Seneca Grad Trip - Karryn Lintelman}

Here's a bit about my Seneca trip, which was exhilarating overall--I had dreams about it all this week!  The first day myself, my leader, Kevin Chartier, and second, Greg Buzulencia, all left the campsite around 9am.  We began with a hike up the stairmaster, and with a short walk we came to Worrell's Thicket, which was a 5.0 climb I think and simple, though I was nervous about the height of it all from the beginning.  This ended up taking us to the Lower and then Upper Broadway ledges, from which we eventually took the east face to Gunsight access, a pitch that Greg led!  At this point we stayed in the notch between the two peaks for a bit, having some lunch and water and conversation.  We even saw a snake slither down the ridge and over our ropes and then beyond.  (Greg saw another snake here do the same thing while he was waiting for his next climb, too.)  Though it was relaxing and the view was great, I was pretty nervous at this point, because the whole time we rested here the Gunsight to South Peak pitch was staring us in the face.  The thin exposed fin was high above us, and this is what I was supposed to climb!  Eventually, Kevin began the lead and I waited nervously for my turn.  When that time came, I just had to go for it, and I made it up the first pitch (mostly by refusing to look down or anywhere else).  Rope drag on my second rope added a bit to my fear, but clipping through and just trying to make it focused my concentration.  My team was also helpful in shouting encouragements!  After we all made it to the ledge, we had one more simple pitch and walk (crawl) along the top ridge to the summit.  Though this climb was easy, my fear of the height was certainly still there, but the summit itself was beautiful and the late afternoon was gorgeous at this point.  After we all made it, we had a short downclimb, and we ended up rappelling from Old Ladies on the east face, and then again through a wooded side of the cliff.  This was a 9-10 hour day, but I felt good at the end.  Afterwards, I never had a beer that tasted so good ;)  

\begin{Figure}
 \centering
 \includegraphics[width=0.95\linewidth]{pics/karryn.jpg}
 \captionof{figure}{Rappelling}
\end{Figure}


The next day we climbed the first pitch of Totem and rappelled down.  We then went over to climb Gert's Grungy Gully to Vegetable Variation, and from here we walked down the wooden steps and through the creek back to the road around 3pm.  These were fun climbs, though the height itself still unnerved me a bit.  Maybe I'll get used to it?!  Overall, this was a great trip and I had a really fun, informative, experienced, safe team!  Thanks!! And yes, I did call my mother ;)

Attached is a pic of Kevin leading the climb to south peak and Greg belaying him.  

See you all soon!     

Karryn

\subsect{ECP Sailing School - Bill Baxter}

Here's a brief summary of the class of ECP Sailing School 2013,

My son Jeff and I each have small sailboats that we hadn't sailed in 2 or 3 years so we decided that if we offered a sailing class we'd have to clean them up and get them ready to sail.   So we finally found dates that would work and sent out the message to the listserve.

We ended up with 5 students.  We had class at my house and we went over sailing terms and had the students help set up both boats.  Then the following Saturday we went off to Moraine State Park to sail.   Only 3 students were available to sail Saturday which worked out well since that was an ideal number for our two small boats.   Each student got to take the helm and see if they could remember enough that was taught to keep from capsizing the boats.  The winds were variable but stronger than predicted and plenty to have a lot of fun.  Several times we brought the two boats together to swap students so each student had a chance to sail in Jeff's faster boat.  Everyone did well.

\begin{Figure}
 \centering
 \includegraphics[width=0.95\linewidth]{pics/baxter_sailing.jpg}
 \captionof{figure}{Bill Baxter Sailing}
\end{Figure}

The next Saturday I went back to Lake Arthur with Lisa (the 5th student couldn't make it).  We had stronger and more consistent wind than the previous week.  There was a sailboat race right in front of the boat launch so we had to try to keep clear of the racing boats as we crossed the lake.  Lisa did well and practiced tacking many times.  We were sailing to the boat ramp when the wind picked up.  I was on the down side of the boat and was getting ready to cross to Lisa's side to balance the stronger wind when an even stronger gust hit us.  Lisa knew that turning the boat into the wind would lessen the tipping.  Well I guess she figured that if turning a little bit into the wind is good, a lot must be better.  She turned the tiller as far as it would go and the boat spun around quickly and the wind hit the sail from the other side and tipped it enough that water poured in over the gunwale onto the seat where I was sitting and filled the cockpit of the boat.  But we didn't capsize and Lisa volunteered to find the bailing bucket and undo her work.  I found out that my boat sails pretty well full of water.  Jeff's boat has a self bailer that sucks the water out when you are moving but mine does not.   Lisa got most of the water out and we headed back with a fun story to tell.

We hope to have another sailing adventure in August.  I'll post it on the listserve in case any of you want to join us.

Captain Bill Baxter

\subsect{ECP Sailing School - Mike Ciccone}
It was a pleasure to take part in the first ECP sailing School.  The class met at 7PM on Tuesday June 11 2013 at Bill Baxter's House in Ross Twp.  Four students arrived for the class session.  The class began on the deck when students received a Boy Scouts of America Handbook on Sailing Small Boats.  A few questions were asked and then we went into the back yard where Bill and Jeff Baxter's Sailboats are kept.  The 6 people split into groups of 3 and we began by uncovering the boats.  Bill's Boat is a bit larger and has room for 4 people, while Jeff has a smaller but faster boat with room for 2 people.  Students were directed through the activities needed to prepare the sailboat for launch.  Masts were set and rigging was attached.  At least one ECP member was a bit disappointed at the absence of a boomvang on either boat.  We were informed that ropes are not called ropes- but instead called lines in sailing.  There are Halliards to raise sails, and sheets to change the tension on sails when underway.  One of the first things we did was open the sails.  On Bill's Boat the sail was neatly folded in a bag (about the size of a tent bag).  On Jeff's Boat the sail furled around the mast, which neatly stored in his boat (fitting into 2 built in holes.  A simile pull on a line was enough to unfurl Jeff's Sail.  Bill's Sail needed to be rolled out and attached to the mast.  At this point it could be raised with the Halliards.  It is interesting to note that rigging for masts attaches to the gunwales of the boats via a small clip that looks like a keychain which fits neatly into a grommet.  Each boat has 2 sails (main sail and Jib Sail).  The sails receive reinforcement with a device called a batten (just a piece of wood or plastic that fits into a slot on the sail.  Once the boats were fully rigged up with masts rising (? 15-20' into the air) we practiced attaching them to hitches at the back of Bill and Jeff's Vehicles.  We discussed sailing terminology- which is quite extensive.  The back is the stern, the front is the bow.  The 2 sides are called starboard and port.  Each boat was equipped with a centerboard- a small rudder that lowers from a horizontal to vertical position from a slot in the middle of the sailboat.  The centerboard is needed to add stability and help with steering the boat.  other types of sailboats can have dagger boards (small removable boards) that are placed in a hole and go straight down into the water, or keels.  Keels are used mostly in larger boats where a large amount of weight is needed below the sailboat to stabilize it.  We learned that there is a "no go zone" of 45 degrees (where the sail boat will not move because no wind is reaching the sails.  Another area where the sailboat will lot move much is when the boat is facing directly into the wind- called irons.  At this point the sails flap back and fourth rapidly- called luffing.  The fastest speeds are on a broad reach.  Time went fast, and before we knew it- darkness was setting in.  At this point we unhitched and derigged the boats to prepare them for storage until our 6/15/2013 voyage at Moraine State Park.  We then went into Bill's house for beverages and to discuss the logistics for Saturday's Trip.  Around 9:35PM the group began to break up to go home.
 
 
 
\begin{Figure}
 \centering
 \includegraphics[width=0.95\linewidth]{pics/mike1.jpg}
 \captionof{figure}{Ciccone Sailing}
\end{Figure}

\begin{Figure}
 \centering
 \includegraphics[width=0.95\linewidth]{pics/mike2.jpg}
 \captionof{figure}{Ciccone Sailing}
\end{Figure}
 
Four days later on Saturday 6/15/2013 three students arrived at Bill Baxter's house at 9AM to go sailing.  Two of the students were not there, however a new student was present.  Around 10AM or so the boats were hitched and we were off to Moraine State Park (South Shore).  Upon arrival- sometime around 11AM- we rigged up the sailboats for sailing in the parking lot, and hauled them to the lake.  We were greeted with a nice variable wind (up to about 12MPH) a bit stronger than the 7MPH predicted by the weather forcasters.  This was kind of unusual since the wind usually picks up in the afternoon at Lake Arthur.  The wind at Lake Arthur is also extremely variable changing direction all over the compass every few minutes.  Very shortly after getting on the lake, the wind filled the sails and we began moving pretty fast.  Wind only lasted a few minutes at most, before we would slow down and wait for the next burst of wind.  You could see the wind coming on the lake by the ripple patterns.  We went across the lake once or twice, and getting back required tacking or sailing back and fourth into the wind (but not directly into the wind).  The morning was cool (in the 60's).  After lunch, however the wind died off completely, and it warmed up to about 80 degrees.  This made for a slow trip, but also made it easier for students to switch between Jeff and Bill's boats when they were together in a flotillia.  After several hours of little or no wind and watching some larger boats with spinnikers (big parachute like sails at the front of the boat) move a little, we decided to head in.  This required that we paddle (yes there are paddles in the sailboats) to shore once we got close.  Around 3PM or 3:30PM (I forget exactly when) we docked the boats and loaded them onto the trailors of Jeff and Bill's vehicles.  We then has some iced tea, and drove back.  Considering the less than optimal wind conditions this class went over very well with students learning a lot of terminology, and all students having the opportunity to steer and sail both boats.  Thanks Bill and Jeff for a great class.


\subsect{ECP Sailing School - Ron Edwards}

Over the past couple of years, Bill Baxter has poked around for possible interest in a sailing school.  
On June 6 I got my acceptance letter into the school.  Yar!
Santosh Chandrasekaran, Bill Brose, Mike Ciccone, Lisa Falenski and I were the official students.  Ahoy mateys!  (Too much, too soon?)

Bill provided an electronic copy of our book along with reading assignments in advance so that we would be prepared for class.  On Tuesday June 11 we gathered at Bill Brose’s house for our hands-on training.  After reviewing the readings and brief Q\&A, we started assembling Bill’s boat, followed by Jeff’s.  Both are similar craft (sloops) with enough differences to get a sense of variety in construction, parts, and purpose.  There are a LOT of terms in sailing (possibly even more than climbing!) and Bill and Jeff put us at ease covering the essential ones but assure we didn’t need to know every one (such as the individual names for each edge and corner of a main sail).  My favorite term is the “boom vang” and I requested everyone call me that for the rest of the class.
\begin{Figure}
 \centering
 \includegraphics[width=0.95\linewidth]{pics/ron1.jpg}
 \captionof{figure}{Ron Sailing}
\end{Figure}


After some hypothetical situations of wind direction and turning and tacking we disassembled the boats and finished up the lecture over some beers (my kind of sport!)
On the following Saturday, June 15, the weather was clear and sunny, though little and low winds were predicted.  Lisa and Bill were not able to attend, but Santosh, Mike, and I were ready to set sail.  Apparently sailing is not an early bird type of sport.  Winds are best mid-day, so it’s usually a casual start, casting off around 10am.  I wouldn’t need a headlamp or even an alarm to be ready for this outing!
We headed to Lake Arthur, part of Moraine State Park, just a short hour drive north of Pittsburgh. There is also nice technical (rocky) mountain biking, trail running, and McConnell’s Mill is nearby for rock climbing.  The 3,225-acre Lake Arthur has 10 public boat launches. Sailing conditions are often ideal, and races, regattas and sailing instruction classes are held throughout the summer.  There is a sailing club at the lake with boats for rent.  Sail boats share the water with motor boats (up to 20HP).
\begin{Figure}
 \centering
 \includegraphics[width=0.95\linewidth]{pics/ron2.jpg}
 \captionof{figure}{Ron Sailing}
\end{Figure}


We pulled into the launch area and quickly got to work assembling the boats as we had learned previously.  Backing the boats into the water we applied our hitch skills to secure them to the cleats of the dock.  We dropped the center boards, donned PFDs, boarded, hoisted the jib and cast off!
The wind was higher than predicted, we guessed between 5 and 13 MPG, and soon under mainsail, we were tacking across the lake.  We soon learned that the lake’s wind can be fickle, pocketed with steadier or calmer winds and being more reliable near the center.  After the basics had been covered and many turns made, we students were given the responsibility of manning the tiller (rudder control).   Amazingly, no other boaters were harmed during our shifts.  We also managed to come alongside each other’s boats twice to jump ships and allow each to experience the other boat.

\begin{Figure}
 \centering
 \includegraphics[width=0.95\linewidth]{pics/ron3.jpg}
 \captionof{figure}{Ron Sailing}
\end{Figure}


By noon the wind had died down significantly and we were drifting more than harnessing wind in our sails.  Lunches came out, sun beat down, and even a nap was had.  Floating in the middle of a calm lake is about as relaxing as it gets.  Eventually we decided to call it a day and used oars to paddle our way back to shore.  Beers aren’t permitted in the state park so we took a short hike into the woods, accidently uncapped some and emptied them appropriately before packing the bottles out.
\begin{Figure}
 \centering
 \includegraphics[width=0.95\linewidth]{pics/ron4.jpg}
 \captionof{figure}{Ron Sailing}
\end{Figure}


There is talk of heading up to Lake Erie in the late summer to sail the bay.  Many thanks to Bill and Jeff for graciously sharing their knowledge, boats, and time to introduce us new sailors to a wonderful activity.
A bit of wisdom, appropriately cast: “A Smooth Sea Never Made a Skillful Sailor”.

\clearpage
\pagebreak

\sect{Club Business}
This sections contains information on club business and other official matters. 

\subsect{Meeting Schedule and Locations}

\Large
\textbf{JUNE GENERAL MEETING}\\
Thursday, June 13, 7:30PM\\
David Lawrence Shelter\\
Schenley Park, Pittsburgh
\\
\\
\textbf{JULY BOG MEETING}\\
Thursday, July 18, 7:30PM\\
Rush Howe's House

\normalsize


\subsect{JUNE GENERAL MEETING AGENDA}
\textbf{OFFICER REPORTS}\\
\textit{\textbf{President, Rush Howe}} -- Announce:
Audit Committee \& Other Appointments\\
\textit{\textbf{Vice President, Bill Baxter}} \\
\textit{\textbf{Secretary, Phil Sidel}}
\\
\\


\textbf{APPOINTEE REPORTS}\\\\
\textit{\textbf{Membership Coordinator, Martha Gray}} : The following have applied for Membership:

\begin{center}
	\begin{tabular}{c}
		Matt Baily
	\end{tabular}
\end{center}


\textbf{BOG MINUTES READING}

\begin{itemize}
\item Rush Howe will set up a family oriented bike ride event in memory of Chip Kamin 
\item Account Balance totaling \$22622.73 as of May 1st was reported and accepted. Details are published in the June Newsletter
\item Plans moved forward for Annual Party/Roast June 1-2 at ACE Resort in WV.
\item Website Officer List and Summer Meeting Announcements were updated.
\item Phil Sidel was to send out email calling for volunteers to set up and manage an ECP table at a Riverside Park event on June 8th. (I neglected to carry this out)
\item Webcasting meetings was delayed until indoor meetings resume
\item Plans for website standards to be implemented was reviewed.  The planned membership database will not be implemented.
\item Some rich trip report materials - especially some published by Ivan Jirak - are not currently in the historical archive
\item Jamie Buillings will again be Backpacking School Committee Chairperson
\item Chris Ciesa and Jessica Goelz will be Directors for 2013.
\end{itemize}

\hrule 

\subsect{MAY GENERAL MEETING MINUTES} 
ECP GENERAL MEETING\\
MAY 9th 2013\\
Forbes \& Braddock Shelter in Frick Park\\
Meeting opened 7:50p.m. \\
27 members \& 3 guests attending
\\
\\
\textbf{OFFICER and APPOINTEE REPORTS}\\
\\
\textit{\textbf{President Rush Howe}}: "We are here - June, July, and August will also be outside". Meeting in June will be held David Lawrence shelter in Schenley Park.
\\
\\
\textit{\textbf{Vice President Bill Baxter:}} Tonight's Program will be a demo of Whisperlite stove maintenance by Ron Edwards.  Next month, Kevin Chartier is presenting on a ski mountaineering trip to the Tetons.\\
\\
\textit{\textbf{Secretary Phil Sidel:}}  April General Meeting Minutes are published in the May Newsletter.. No BOG meeting was held last month. Thanks Lisa Falenski for taking notes at this meeting - as a basis for minutes to be published.Since Irene will be unable to be at the May BOG meeting,  Phil needs a volunteer to take notes at that meeting.  Any Volunteers?  None raised a hand \\
\\
\textit{\textbf{Librarian Phil Braedenbach:}} Not Present - still looking for help on Library - someone to attend meetings.\\
\\
\textit{\textbf{Membership Coordinator Martha Gray:}}  Nominated following applicants  for new membership:

\begin{center}
	\begin{tabular}{c}
		Jonathan Seethaler\\
		Kate Ekmann\\
		Erik Zambelli\\
		Peter Erin\\
		Ben Pizii
	\end{tabular}
\end{center}

Nominations were duly seconded and all applicants were unanimously elected to membership. Two of the applicants were present and spoke of their reasons for joining ECP.\\
\\
\textit{\textbf{Activities Chairperson Ron Edwards: }} 
\begin{itemize}
\item Announced scheduled ECP Activities pre-meeting rides before the summer general membership meetings - keep eyes posted on list-serv for details.
\item Rock Climbing School Events.  'Final "tests" this weekend at White Rocks. Students will soon be coordinating Tuesday evening climbs at McConnell's Mill,   details to be announced on list-serv.
\item Annual Party at ACE Adventure Resort in New River Area WV  Weekend of July 1-2 - Go online now to buy tickets  Proceeds to benefit Mike Brown Memorial Fund. Variety of events planned including raft trips both Saturday and Sunday; possible zip line event, lots of rock climbing and hiking. Also will be a gear auction benefit of MBEG fund
(donate you surplus gear).
\item June 7-9 - Rock School "Graduation Climb" at Seneca - Volunteer Leaders and Experienced Seconds always in demand.
\item Annual "Seneca Summit" for women climbers to upgrade their skills.- At end of June
\item MS150 charity bike ride for Maggie's Marauders will be 7/20-7/21.  John Zolko is soliciting for participants. They're holding practice rides at North Park each Weds. at 5:00 p.m. Each rider must raise \$300.00. Maggie's group reserves rooms in Altoona on Friday, 7/19.  75 mile ride first day and 75 miles back following day. Staying overnight at Penn State campus.
\item July 27-28 - Flotilla - down the Allegheny or one of its tributaries.
\item Other events sponsored by other organizations are posted on the website events calendar
Watch the ECP List-serv for announcements of other, (short term) activities
\item Rush Howe announced  Mountain Biking at Bavington next weekend
\item John Zolko announced an urban climb on St.Peter and Paul church in East Liberty steeple to repair a damaged roof.
\item Bill Baxter announced a small boat sailing class -  probably classroom session June 11, with on-hands sailing on Lake Arthur  (Moraine State Park) the following two weekends. Details to follow.
\item Mike Ciccone announced plans for a Memorial Weekend Climb
\item Scott Ross announced Biking at Raystown Lake next weekend.
\end{itemize}

\textbf{OLD BUSINESS}\\
Derek Stuart nominated and elected as Equipment Chairperson(Gear Guy), replacing Paul Guarino who had to resign since he is moving into smaller quarters. (The gear has already been transferred to Derek's house)
\\
\\
\textbf{NEW BUSINESS}\\
Honoring Chip Kamin :   \\
Family Bike Ride in North Park - proposed - date to be finalized. Can ride in or donate to Maggie's Marauders/MS 150.\\
\\
Tee-Shirts Sale: Martha Gray announced that they found extra t-shirts from 2010/2011 mountaineering school. Selling for \$5.00 each with proceeds to go to mountaineering school.\\\\1
No Raffle planned for this meeting\\
\\
\textit{Adjourned  8:17 p.m. upon motion duly made and seconded}
\\
\\
\textbf{PROGRAM}\\
Ron Edwards demonstrated the disassembly, maintenance, and reassembly of Whisperlite Stoves.

\hrule 

\subsect{MAY BOG MEETING MINUTES}
ECP BOG MEETING\\
MAY 16th 2013\\
Location: Kathleen Prigg's home\\
Attendees:  Rush Howe, Bill Baxter, Ron Edwards, Kathleen Prigg,
                    Martha Gray, Phil Sidel,  Tom Prigg, Logan Prigg\\
Meeting opened  7:51 pm\\
\\
\textbf{OFFICER REPORTS:}
\textit{\textbf{President, Rush Howe:}} will set up and lead a family-oriented bike riding event in North Park in memory of Chip Kamin - date and details to be determined.\\
\\
\textit{\textbf{Vice President, Bill Baxter:}} June meeting slide show at Schenley will be on back-country skiing in the Tetons, presented by Kevin Chartier.\\
\\
\textit{\textbf{Secretary, Phil Sidel:}}  Requests that people read the draft minutes of meetings they have attended and notfy him of corrections to be made before the minutes are published in the newsletter.
Thanks Martha Gray for taking notes on this meeting.  Announces that  since Irene will be unable to attend meetings for the foreseeable future,  he will need volunteer note- takers at upcoming meetings - probably for the rest of the year.\\
\\
\textit{\textbf{Treasurer Kathy Prigg:}} Submitted a printed report of the financial state of he club as of May 1st:   
\begin{center}
	\begin{tabular}{l l}
	   	Total Balance &  	\$22,622.73 \\ 
	Bogel Fund	&	\$10,084.68\\
	MBEG Fund	& \$2,266.55\\
	Equipment Fund & \$1,445.84\\
	General Fund	 & \$8,825.66 
	\end{tabular}
\end{center}

The detailed table of Revenues and Expenditures - compared with the 2013 budget - will be printed in the June Newsletter. Kathy will also send a copy of the 2012 Audit Committee Report to the BOG \& Editor for publication in the June Newsletter.\\
\\
\textit{\textbf{Activities Chairperson, Ron Edwards:}}  Reviewed upcoming events\\
\begin{itemize}
\item Annual Party at ACE Resort, WV. Ron will send reminder for people to buy tickets
\item Using  3rd Party App to be reviewed
\item Club is soliciting donations of gear to be auctioned - proceeds to MBEG
\item Kathy will be in charge of dinner for Saturday night
\item Looking for volunteer to provide program for Saturday Night
\item Don Wargowski and other ECPers at ACE Resort are actively working on planning and implementing the event and are looking for help 
\item Other "long term" events are posted on Activities Calendar - Anyone can post  an event; Ron reviews the postings.  Meetings are not posted on the Calendar.  
\item "Short term" events (e.g. "next weekend" trips) are announced on list-serv, not posted on the Calendar.\item Website updates needed:  
\begin{itemize}
\item  New Equipment Chairperson needs corrected listing
\item Summer Meeting Locations should be shown.
\item Ron made the updates during the meeting.  COMPLETED.
\end{itemize}
\end{itemize}

\textit{\textbf{Equipment Chairperson, Derek Stuart}} :  (voted in at May General Meeting)
  Not present, no report\\
  \\
\textit{\textbf{Editor, Deep Moitra:}} Not Present; Leaving for long visit to his home in India.
  Phil Sidel will email him to check that he can publish newsletter while in India.\\
  \\
\textbf{APPOINTEE REPORTS:}\\
\textit{\textbf{Membership Coordinator, Martha Gray}}
Martha sent out email to members with directions on how to join ECP list-serv. Seven people joined.\\
\\
\textit{\textbf{Advertising Chairperson, Tara Powers}} :  Tara is stepping down; need new one.Phil received phone message inquiring if ECP is interested in setting up a table at a Riverview Park event the second saturday in June - the event is approximately 4 hours long.  Phil will send out an email for volunteers to represent ECP at that event.\\
\\
\textit{\textbf{Webmaster, Tom George:}}Plan for Website has been reviewed by Tom, Tony, and Tina.  Nine items left to be done; Those three people will help with remaining items. The planned member data base is deemed too complex a task, so the membership database will be maintained off-line, as it was before the new website was established.  This means that we will not have any members-only  sections of our website.\\
\\
\textit{\textbf{Environmental Chairperson, Ginette Vinski:}} Not present; no report \\
\\
\textit{\textbf{Historian, Phil Sidel:}} Nothing to report.\\
Later, Tom Prigg asked if minutes for 1967-1968 were available.  Answer was "Yes, they are available, but pulling them together (extracting from newsletters) would take quite a bit of time.  Tom also asked if the archives included any Ivan Jirak trip reports.Phil responded that there were a large number of Ivan Jirak items - not yet cataloged in detail, but he didn't recall any "Trip Reports" of the sort Tom was describing,.  Tom showed some of the items he had collected from John Timo and others, including two trip reports in booklet form.  Phil noted that he had not seen any booklet trip reports like that in the archives.
BOG also discussed potential of engaging an archivist to perform a bulk scanning of all paper archives, including a box Tom has from the early years of the club, including impressively detailed and bound trip reports.\\
Next action:  Tom and/or Ron to seek potential services.
\\
\\
\textit{\textbf{Policy Revision Committee:}}
Tom Prigg will draft MBEG Policy Statements for review by the BOG and Committee (any interested members).  Hopefully meet on it in June.  The main part of the MBEG policy is posted on the website, but not sections about its funding from other club activities, nor various adjustments to the application form and procedure that were discussed when the first application was submitted.
\\
\\
\textbf{OLD BUSINESS:}\\
\textbf{Summer Meetings:}
\begin{itemize}
\item Rush will buy food for June Meeting
\item Martha will buy food for July Meetng
\item Bill will buy food for the August meeting
\item Currently Phil Sidel has the reservation for the June Meeting. It is a "No Alcohol" reservation;  If alcohol (beer) Is to be included, someone willhave to get a revised reservation (\$75 additional fee) and be the reservation holder for the event.  BOG decided to revise the reservation,  Ron Edwards will obtain the 
  revised reservation permit.
\end{itemize}
Webcasting Meetings - Delayed until indoor meetngs resume.\\
List-Serv - Martha Gray has been made a Moderator.
\\
\\
\textbf{Annual Tasks for May:}
\begin{itemize}
\item Mountain School Chairmanship:  Felix Duvalet officially appointed for 3rd term
\item Flag and Life Membership Qualifications were published in May Newsletter.
\item A proposed list of environmental organizations to which ECP should contribute or Join as an organizational member should be presented.  Kathy will review last year's BOG minutes to get list of donations to be made for 2012 and proposed list for 2013.
\item Phil noted that Rachel Carson Trails Council  has published its financial statement for 2012, and very little of the income has been spent on program.
\item Backpacking School Committee/Chair:  Jamie Billings will again serve as Committee Chairperson;  Chris Ciesa \& Jessica Goelz  will be 2013 Directors..
\end{itemize}

\textbf{NEW BUSINESS:}
\begin{itemize}
\item Sailing School - Dates TBD 
\item Future BOG Meetings. No June meeting Planned
\item Rush Howe will host July Meeting
\item Eventbrite Software working well
\item Chip Kamin Memorial Bike Ride - Family event:  Plan to include Kids' Ride, Mt. Bike Ride, Road Ride.It will be a fundraiser for Maggie's Marauders. Would include Potluck Picnic, 50/50 Raffle. Need Location (shelter).Date TBD;  Thinking of a Sunday in September
\end{itemize}

\textit{Adjourned 9:40p.m.:}

\hrule 

\subsect{Audit Committee Report:}

\textbf{Audit meeting/participants:}  On March 11, 2013, the audit was conducted of 2012.  Kathy Prigg, who took over the position of Treasurer mid-year in 2012 and is also the 2013 Treasurer presented the records as both the outgoing and incoming Treasurer.  Valerie Kramer, Paul Toth and Martha Gray comprised the Audit Committee.
\\
\\
\textbf{Audit Summary:}  The ECP records are kept in an Excel spreadsheet which is working well as a mechanism to track expenses and allocate revenues.  It also provides a simple way to compare budget to actuals. The Committee reviewed the detailed transactions for three months, selected randomly, confirmed bank balances were reconciled and reviewed the process of allocating monies to the various ECP funds (General, Bogel, etc.).  Everything was found to be in order and Kathy is going a great job of tracking things and making the necessary detailed entries.  The following is a summary of items discussed:

\begin{itemize}
\item The ability to pay dues via PayPal is working well.  The Treasurer is able to pull funds and reconcile them.  There are some additional small amounts due to how PayPal assesses fees and the Treasurer has been including them in the General Fund which the Committee concurs is correct.

\item Due to the limited number of budget lines, the Treasurer has been making notes attached to specific Excel cells if a given dollar amounts represents multiple items - this is a great practice and very helpful.

\item The Treasurer suggested that it would be a good idea to have cash receipt forms and to be able to give receipts when people give her cash - such as for memberships or for the 50/50 raffle.  The committee agreed this was a good idea and a good practice.

\item Previously, although the Mike Brown Memorial Fund had been authorized by the club, the financial spreadsheet and budget was not recording the revenue allocated to the Fund.  The Treasurer added this into the ECP financial records and has been accurately accounting for the Mike Brown Memorial Fund throughout 2012.  

\item The Treasurer is working on establishing a different money market account with a better return than the current PNC account; the committee supports this effort.
For record-keeping purposed, all receipts turned in should be marked as to who is submitting them and what is the purpose - can be handwritten on the receipt such as, "Bill Explorer for BOG meeting food - 3/4/2013.  The Treasurer can add a note as to the check number and date paid.


\item Contributions to various organizations were approved by BOG and membership during 2012 and should be made.


\item There are currently lines in the ECP General Fund Revenue and Expenses for the various schools.  We recommend those be eliminated as there is no budget allocation.  Any revenues and expenses should be recorded in the lines for the schools set up under the Equipment Fund.  The duplicate sets of lines for the same items cause confusion.  The Treasurer will eliminate these for 2013 based on the Committee's recommendation.  This will have no impact on the budget or the allocation or accounting for the schools.

\item Current practice is that school leaders are given the school tuition (other than that allocated to funds such as Bogel or Mike Brown) in advance for school related expenses.  Since the school tuition checks are written to the ECP, they are deposited by the Treasurer and then a check is written to the school leader for funds available for school expenses.  Gear deposits are left in the ECP accounts until the school refunds them - the refund process is up to the school directors.  The Audit Committee recommends at the end of each school, the school instructors should submit an accounting of what was spent as support for the initial allocation of funds.  Our understanding is that this is typically part of the final school report as well.  Any unused funds would be returned to the club.  Ron Edwards used a spreadsheet that we would recommend all school instructors use for this purpose.   


\item The overall accounting seems more complex than is necessary.   We suggested that there is an opportunity to simplify things if someone was interested in addressing this but it is not within the scope of the Treasurer's job (unless desired).  Some suggestions might be to have less line items (not track family and individual memberships), automate the splitting of dues and listing all schools in one place only.

\item While this is not in the scope of the audit, there was some discussion about membership with input from Martha in her role as the new Membership Chair and feedback she received from the current Membership Chair.  This committee would support efforts to remind current members who do not renew that a renewal is due. As an example, members of the 2012/13 Mountaineering School may not realize that though they joined in the fall, a renewal is due by December for the coming year.   Also, the ECP may want to consider letting new members join and pay online with their approval being done at the following meeting.  It is rare for a member to be rejected and it appears potential members are lost when they can't join online.  It is a barrier to require a paper application. Other items that came up is that PayPal may offer an automatic renewal and PopMoney may be a better or an additional way to collect funds.
\\
\\
In general, we want to thank Kathy for her diligence and her overall efforts - both in picking up the Treasurer job mid-year and continuing through 2013.  We also want to thank her for sitting through a long meeting while some of us acquainted or re-acquainted ourselves with ECP practices!
\\
\\
Respectfully submitted - \\
Valerie Kramer\\
Martha Gray\\
Paul Toth 
\end{itemize}


\end{multicols}
\clearpage
\pagebreak

%\subsect{Treasurer Report}
%
%\begin{Figure}
% \centering
% \includegraphics[width=\linewidth]{pics/treasure1.pdf}
%\end{Figure}
%
%
%\begin{Figure}
% \centering
% \includegraphics[width=\linewidth]{pics/treasure2.pdf}
%\end{Figure}

\pagebreak

\subsect{Other Announcements}

\begin{Figure}
 \centering
 \includegraphics[width=\linewidth]{pics/extrav.pdf}
\end{Figure}


\pagebreak
\clearpage



\appendix

\section{Contact Information}

\subsection{Activities Coordinators}
For some of our activities, we currently have no active coordinators. If you think you might be able to coordinate
and lead these activities, contact Activities Chairperson Ron Edwards ecp@edwardsjr.com (412-327-2084)


\begin{center}
    \begin{tabular}{ | l | l | l | }
    \hline
    \textbf{Activity} & \textbf{Contact} & \textbf{Email} \\\hline
	Backpacking & Suraj Joseph & surajj1234@gmail.com \\ \hline
	Biking - Mountain & Rush Howe & rushhowe@yahoo.com \\ \hline
	Biking - Road & Chip Kamin & chip.kamin@gmail.com \\ \hline
    Caving  & Doug Fulton & fulton12b@yahoo.com\\ \hline
    Fly Fishing & Bruce Cox & brcox33@comcast.net\\ \hline
    Ice Climbing & Tom Prigg & Tom.prigg@gmail.com\\ \hline
    In-Line Skating & Robin Kamin & ekamin@verizon.net\\ \hline
    Mountaineering & Sam Taggart & samuel.taggart@gmail.com\\ \hline
    Paddling - Flat Water & Nick Ross & Lnickross@gmail.com\\ \hline
    Paddling - White Water & Barry Adams & bj2adams@juno.com\\ \hline
    Rock Climbing & Ron Edwards & ecp@edwardsjr.com\\ \hline
    Rowing & Bob Dezort & bobdezort@verizon.net\\ \hline
    Trail Running & Brian Ottinger &  brian-ecp@ottinger10.com\\ \hline
    Sailing & Bill Baxter & billybax@yahoo.com\\ \hline
		   &  Jeff Baxter& jeffreywbaxter@yahoo.com\\ \hline
	Skiing & Downhill  &  Lindsay Hastings LMH239@gmail.com\\ \hline
	Yoga  &  Allison Pochapin & a.pochapin@gmail.com\\ \hline
            &  Elise Nolan & elise.nolan@gmail.com\\ \hline
	Adventure Racing & & \\ \hline
	Skiing  Cross-Country & & \\ \hline
	Rafting & & \\ \hline
	SCUBA Diving & & \\ \hline
	Sea Kayaking & & \\ \hline
	Triathlon Training & & \\ \hline
	Orienteering (proposed) & & \\ \hline
	Geocaching (proposed) & & \\ \hline
	Skydiving (proposed) & & \\ \hline
	 \hline
    \end{tabular}
\end{center}

\paragraph{What does an Activity Coordinator do?}
General advocacy and point of contact for the activity by helping current, new, or prospective members
get connected with others in the club also interested in the activity and any events planned for it.
\begin{enumerate}
\item Seek out new members at general meetings (during the break or at the after-meeting) that
indicate the activity as one of their interests and introduce yourself as that activities
"coordinator" that can help them connect with and explore the activity.
\item Answer incoming emails to the club (sometimes redirected from officers or others who receive
the inquiry) related to the activity. This is rare.
\item Spread the word if you see any local events or news related to the activity that you think
members would like to know by posting (or reposting) on the Listserv and mentioning at any
General Meeting that you attend. In some cases, create an Event (takes 2 minutes) on the ECP
calendar to give that event extra visibility.
\item Encourage any members that are planning an event on their own or as a small/closed group to
consider opening it up to the general membership in the form of an official Event (and have
them create the Event)
\item If possible, directly or indirectly ensure that at least one Event for your activity occurs every year.
\end{enumerate}

\fbox{\parbox{\linewidth}{
{\Large\bf ECPers!!}\\
Contact the Coordinator for activities in which you want to participate.\\
There are many activities which are not listed in the calendar,
because the arrangements are spur-of-the-moment, or because\\
they are regular, periodic (e.g., weekly) activities.\\
\\
Some examples are:\\
Weekly In-Line Skating Sessions\\
(Contact Coordinator Robin Kamin)\\
Yoga Sessions in Frick Park\\
(Contact Yoga Coordinator Elise Nolan)\\
Flat Water Canoeing trips\\
(Contact Coordinator Nick Ross)\\
Mountain Biking Runs\\
(Contact Coordinator Rush Howe)\\
And many others\dots}}

\subsection{Officers}
\begin{center}
    \begin{tabular}{ | l | l | l | l | }
    \hline
    \textbf{Position} & \textbf{Name} & \textbf{Email} & \textbf{Phone} \\\hline
	President & Rush Howe & president@pittecp.org & 412-983-5265 \\\hline
	Vice-President & Bill Baxter & vicepresident@pittecp.org & 412-926-8261 \\\hline
	Secretary & Phil Sidel & sidel.climbing@gmail.com & 412-521-9570 \\\hline
	Treasurer & Kathleen Dehrer & treasurer@pittecp.org & \\\hline
	Activities Chair & Ron Edwards & activities@pittecp.org & 412-327-2084 \\\hline
	Equipment Chair & Derek Stuart & equipmentchair@pittecp.org & 585-727-1258 \\\hline
	Editor & Subhodeep Moitra & editor@pittecp.org & 412-721-5305 \\\hline
    \end{tabular}
\end{center}

\subsection{Appointees}
The ECP Appointees are persons appointed by the president to fill key positions in the club. In
addition there are appointed Activity Coordinators and Special Committees.

\begin{center}
    \begin{tabular}{ | l | l | l | l | }
    \hline
    \textbf{Position} & \textbf{Name} & \textbf{Email} & \textbf{Phone} \\\hline
	Advertising & Tara Powers &	tarasmagicalpowers@gmail.com &  \\\hline
	Environmental & Ginette Vinski & ginette@vinski.net & 412-366-4925 \\\hline
	Historian & Phil Sidel & sidel.climbing@gmail.com & 	412-521-9570 \\\hline
	Librarian & Phil Breidenbach & Booksandmaps@yahoo.com & 412-486-1450 \\\hline
	Membership & Martha Gray & 	graymf@gmail.com &  \\\hline
	Webmaster & Tom George & webmaster@pittecp.org & 412-831-4711 \\\hline
	\end{tabular}
\end{center}



\pagebreak
\clearpage


%----
\end{document} 

