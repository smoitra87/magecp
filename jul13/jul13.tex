%%% LaTeX Template: Newsletter
%%%
%%% Source: http://www.howtotex.com/
%%% Feel free to distribute this template, but please keep the referal to HowToTeX.com.
%%% Date: September 2011


%%% ---------------
%%% PREAMBLE
%%% ---------------
\documentclass[10pt,a4paper]{article}

% Define geometry (without using the geometry package)
\setlength\topmargin{-48pt}
\setlength\headheight{0pt}
\setlength\headsep{25pt}
\setlength\marginparwidth{-20pt}
\setlength\textwidth{7.0in}
\setlength\textheight{9.5in}
\setlength\oddsidemargin{-30pt}
\setlength\evensidemargin{-30pt}

\frenchspacing						% better looking spacing

% Call packages we'll need
\usepackage[english]{babel}			% english
\usepackage{graphicx}				% images
\usepackage{amssymb,amsmath}		% math
\usepackage{multicol,caption}				% three-column layout
\usepackage{url}					% clickable links
\usepackage{marvosym}				% symbols
\usepackage{wrapfig}				% wrapping text around figures
\usepackage[T1]{fontenc}			% font encoding
\usepackage{charter} 				% Charter font for main content
\usepackage{blindtext}				% dummy text
\usepackage{datetime}				% custom date
	\newdateformat{mydate}{\monthname[\THEMONTH] \THEYEAR}
\usepackage[colorlinks=false]{hyperref}	% links and pdf behaviour
\usepackage{hyperref}



% Customize (header and) footer
\usepackage{fancyhdr}
\pagestyle{fancy}
\fancyhf{}
\lfoot{}
%\lfoot{	\footnotesize 
%		Newletter from HowToTeX.com \\
%		\Mundus\ \href{http://www.howtotex.com}{HowToTeX.com}	\quad
%		\Telefon\ 555-5555											\quad
%		\Letter\ \href{mailto:frits@howtotex.com}{frits@howtotex.com}
%	  }
\cfoot{}
\rfoot{\footnotesize ~\\ Page \thepage}
\renewcommand{\headrulewidth}{0.0pt}	% no bar on top of page
\renewcommand{\footrulewidth}{0.4pt}	% bar on bottom of page

%%% ---------------
%%% DEFINITIONS
%%% ---------------

% Define separators
\newcommand{\HorRule}[1]{\noindent\rule{\linewidth}{#1}} % Creating a horizontal rule
\newcommand{\SepRule}{\noindent							 % Creating a separator
						\begin{center}
							\rule{250pt}{1pt}
						\end{center}
						}						

% Define Title en News input
\newcommand{\JournalName}[1]{%
		\begin{center}	
			\Huge \usefont{T1}{augie}{m}{n}
			#1%
		\end{center}	
		\par \normalsize \normalfont}
		
\newcommand{\JournalIssue}[1]{%
		\hfill \textsc{\mydate \today, No #1}
		\par \normalsize \normalfont}

\newcommand{\NewsItem}[1]{%
		\usefont{T1}{augie}{m}{n} 	
		\large \bf #1 \vspace{4pt}
		\par \normalsize \normalfont}
		
\newcommand{\NewsAuthor}[1]{%
			\hfill by \textsc{#1} \vspace{4pt}
			\par \normalfont}		

\newcommand\sect[1]{%
  \section*{#1}%
  \addcontentsline{toc}{section}{#1}}

\newcommand\subsect[1]{%
  \subsection*{#1}%
  \addcontentsline{toc}{subsection}{#1}}


\newcommand{\HRule}{\rule{\linewidth}{0.5mm}}


\newenvironment{Figure}
  {\par\medskip\noindent\minipage{\linewidth}}
  {\endminipage\par\medskip}

%%% ---------------
%%% BEGIN DOCUMENT
%%% ---------------
\begin{document}



\begin{titlepage}

\begin{center}


% Upper part of the page
\includegraphics[width=0.5\textwidth]{pics/ecplogo.jpg}\\[1cm]    

\HRule \\[0.4cm]
{ \Huge \bfseries THE EXPLORER}\\[0.4cm]

\HRule \\[1.5cm]


\textsc{\LARGE July 2013}\\[9cm]

\textsc{\large The EXPLORER is the monthly newsletter of the Explorers Club Of Pittsburgh,Inc., a non-profit organization devoted to research, adventure and education in outdoor and wilderness recreation and conservation}\\[0.5cm]


% Title

% Author and supervisor
%\begin{minipage}{0.4\textwidth}
%\begin{flushleft} \large
%\emph{Author:}\\
%John \textsc{Smith}
%\end{flushleft}
%\end{minipage}
%\begin{minipage}{0.4\textwidth}
%\begin{flushright} \large
%\emph{Supervisor:} \\
%Dr.~Mark \textsc{Brown}
%\end{flushright}
%\end{minipage}

\vfill

% Bottom of the page
%{\large \today}

\end{center}

\end{titlepage}


% Title	
% -----
\JournalIssue{1}
\JournalName{The EXPLORER}
\noindent\HorRule{3pt} \\[-0.75\baselineskip]
\HorRule{1pt}
% -----

\tableofcontents

\clearpage


% Front article
% -----
\vspace{0.5cm}
	\SepRule
\vspace{0.5cm}



\begin{center}
\begin{minipage}[h]{0.8\linewidth}
	\begin{wrapfigure}{l}{0.41\textwidth}
		\includegraphics[width=0.6\linewidth]{pics/me.jpg}
		\\% this spacer is needed to make the text on the right fit OK
	\end{wrapfigure}
	
	\NewsItem{Message from the editor}

	\emph{Greetings, ECP!} Welcome to the July edition of the club newsletter. In this edition we have several trip reports from the recent Rock School graduates. It seems like they had quite an adventure at Seneca Rocks. Read all about it. The first ever ECP Sailing School was organized last month. Ron Edwards, Mike Ciccone and Capt. Bill Baxter describe their experiences. Club business pertaining to the past general meetings are documented. Agenda for the upcoming July general meeting are presented. Our club Librarian, Phil has acquired a new set of books and has provided a really nice summary of them. Upcoming events of interest are advertised at the end of the magazine. Finally, contact information of the officers and activity coordinators is provided in the appendix. 
\\
\\
-- Subhodeep Moitra (Deep)

\vspace{0.5cm}



	Again as a reminder, you are invited to attend the club general meeting as well as the BOG meeting. The location and date for the General and the BOG meeting is as below.
	
\vspace{1cm}

\begin{multicols}{2}
\Large


JULY GENERAL MEETING\\
Thursday, July 11, 7:30PM\\
Frick Park Pavillion\\
Forbes and Braddock Ave\\
\\
JULY BOG MEETING\\
Thursday, July 18, 7:30PM\\
Rush Howe's Residence \\ 


\normalsize
\end{multicols}
	
\end{minipage}
\end{center}
% -----

\pagebreak
\clearpage


% Other news (1)
% -----
\vspace{0.5cm}
	\SepRule
\vspace{0.5cm}
\begin{multicols}{2}


\sect{Trip Reports}


\subsect{Seneca Grad Trip - Megan Olson}

We had an amazing weekend. Maybe I'm easy to please, but I enjoyed every second! As per my request, we didn't do do anything technical but just had a relaxing time and in the process got a great tour of the rocks. On Saturday, my leader, Brian Dunlavey, my second, Brian Kent ("Brian to Brian"), and I got to do the awesome Stairmaster and then went up Old Ladies.
\begin{Figure}
 \centering
 \includegraphics[width=0.95\linewidth]{pics/meg1.jpg}
 \captionof{figure}{At the base of the climb with Brian Dunlavey}
\end{Figure}
We then did a little walking and a pitch or two to get up to the summit, which was even better than I could have expected. So peaceful (miraculously we were the only people up there at the time) - sun shining, light breeze... Heaven. After making my two Brians pose for photos; I'm sure they didn't want to take (but for which they were good sports and smiled anyway), we all stood shoulder-to-shoulder on a teeny ledge and rapped down the back (east side) of the summit. Yes, I know everyone thinks I'm crazy because I like to rappel, but seriously - this was amazing! We then went up upper Broadway (I'm pretty sure) to the Gunsight notch in hopes of getting up the climb before the end of the day. On the way up we did one of the more "exposed" pieces of climbing I experienced all weekend (although it was just a few moves) - kind of out on the edge and face of a fin w/ nothing below but air. Pretty stunning.
\begin{Figure}
 \centering
 \includegraphics[width=0.95\linewidth]{pics/meg3.jpg}
 \captionof{figure}{Rappelling}
\end{Figure}
Unfortunately when we got to Gunsight there was one group on and another waiting, and since it was late in the afternoon we decided we needed to pack it in. We set up another rappel off a tree near the Gunsight notch, and I again enjoyed myself, yes.
\\
\\
Sunday we were up early again (I think I was the only one who hadn't drunk too much beer the night before? and decided to skip the Stairmaster (we didn't want to spoil ourselves w/ too much of a good thing) to do Skyline Traverse instead. This was an easy but fun climb w/ good variety, including the moment you stem out between the main rock and a pillar and are standing over thin air.
\\
\\
Similar to the day before, we worked our way around lower/upper Broadway, doing a pitch here, a pitch there (including a crack that we're pretty sure no one has touched in 50 years, maybe more?) to again get to Gunsight. This time there wasn't a group climbing, but we were past our decided turn-around time, so Gunsight will have to wait for another day.
\\
\\
I really had no expectations for the weekend (mostly b/c I knew nothing about Seneca and decided to wait until I saw it in person to find out), but it was definitely cooler than I could have imagined. Brian \#1 was an awesome leader, patiently and calmly making sure we were always safe; and he taught me a lot - even simple stuff like how you logistically get three people up and around a mountain safely w/ your gear, rope management, history of climbing and of the Seneca area, etc. It's a great feeling to be totally comfortable 1,000 feet off the ground b/c you trust who you're w/ and you know they're knowledgeable and are looking out for you. So great job, Brian to Brian!
\\
\\
Annnnnd the fact that I got to go swimming for the first time this summer in a crystal clear river and saw a really cute (albeit possibly poisonous) snake only made the weekend that much better. Plus, check out the weather we had... Unreal. What a weekend! Thanks to everyone who made the trip possible (esp. Paul, Matt and my team) - I know organizing that many people to do something as "serious" as climbing is no small feat, but I'm so grateful you are willing to do it!
\\
\\
-- Megan Olson Hunt
\begin{Figure}
 \centering
 \includegraphics[width=0.95\linewidth]{pics/meg2.jpg}
 \captionof{figure}{At the top with Brian Kent}
\end{Figure}


\hrule

\subsect{Seneca Grad Trip - Karryn Lintelman}
Here's a bit about my Seneca trip, which was exhilarating overall--I had dreams about it all this week!  The first day myself, my leader, Kevin Chartier, and second, Greg Buzulencia, all left the campsite around 9am.  We began with a hike up the stairmaster and with a short walk we came to Worrell's Thicket. It was a 5.0 climb I think and simple.I was nervous about the height of it all from the beginning.  This ended up taking us to the Lower and then Upper Broadway ledges. From there we eventually took the east face to Gunsight access, a pitch that Greg led!  At this point we stayed in the notch between the two peaks for a bit, having some lunch and water and conversation.  We even saw a snake slither down the ridge and over our ropes and then beyond.  (Greg saw another snake here do the same thing while he was waiting for his next climb, too.)  
\\
\\
Though it was relaxing and the view was great, I was pretty nervous at this point, because the whole time we rested here the Gunsight to South Peak pitch was staring us in the face.  The thin exposed fin was high above us, and this is what I was supposed to climb!  Eventually, Kevin began the lead and I waited nervously for my turn.  When that time came, I just had to go for it, and I made it up the first pitch (mostly by refusing to look down or anywhere else).  Rope drag on my second rope added a bit to my fear, but clipping through and just trying to make it focused my concentration.  My team was also helpful in shouting encouragements!  After we all made it to the ledge, we had one more simple pitch and walk (crawl) along the top ridge to the summit.  Though this climb was easy, my fear of the height was certainly still there. Nevertheless, the summit itself was beautiful and the late afternoon was gorgeous at this point.  After we all made it, we had a short downclimb and we ended up rappelling from Old Ladies on the east face. Then again through a wooded side of the cliff.  This was a 9-10 hour day, but I felt good at the end.  Afterwards, I never had a beer that tasted so good ;)  
\begin{Figure}
 \centering
 \includegraphics[width=0.95\linewidth]{pics/karryn.jpg}
 \captionof{figure}{Kevin on Lead, Greg on Belay}
\end{Figure}
The next day we climbed the first pitch of Totem and rappelled down.  We then went over to climb Gert's Grungy Gully to Vegetable Variation, and from here we walked down the wooden steps and through the creek back to the road around 3pm.  These were fun climbs, though the height itself still unnerved me a bit.  Maybe I'll get used to it?!  Overall, this was a great trip and I had a really fun, informative, experienced, safe team!  Thanks!! And yes, I did call my mother ;)
\\
\\
-- Karryn Lintelman

\hrule
\subsect{ECP Sailing School - Bill Baxter}

Here's a brief summary of the class of ECP Sailing School 2013. My son Jeff and I each have small sailboats that we hadn't sailed in 2 or 3 years so we decided that if we offered a sailing class we'd have to clean them up and get them ready to sail.   So we finally found dates that would work and sent out the message to the listserve.
\\
\\
We ended up with 5 students.  We had class at my house and we went over sailing terms and had the students help set up both boats.  Then the following Saturday we went off to Moraine State Park to sail.   Only 3 students were available to sail Saturday which worked out well since that was an ideal number for our two small boats.   Each student got to take the helm and see if they could remember enough that was taught to keep from capsizing the boats.  The winds were variable but stronger than predicted and plenty to have a lot of fun.  Several times we brought the two boats together to swap students so each student had a chance to sail in Jeff's faster boat.  Everyone did well.

\begin{Figure}
 \centering
 \includegraphics[width=0.95\linewidth]{pics/baxter_sailing.jpg}
 \captionof{figure}{Bill Baxter Sailing}
\end{Figure}

The next Saturday I went back to Lake Arthur with Lisa (the 5th student couldn't make it).  We had stronger and more consistent wind than the previous week.  There was a sailboat race right in front of the boat launch so we had to try to keep clear of the racing boats as we crossed the lake.  Lisa did well and practiced tacking many times.  We were sailing to the boat ramp when the wind picked up .  I was on the down side of the boat and was getting ready to cross to Lisa's side to balance the stronger wind when an even stronger gust hit us.  Lisa knew that turning the boat into the wind would lessen the tipping.  Well I guess she figured that if turning a little bit into the wind is good, a lot must be better \footnote{Lisa : My turning the tiller as far as it would go was completely intentional. I figured, what's a good introduction to sailing without almost toppling the boat? It was thoroughly refreshing,anyway. Thank you, Bill and Jeff, for your generous instruction and time!}.  She turned the tiller as far as it would go and the boat spun around quickly and the wind hit the sail from the other side and tipped it enough that water poured in over the gunwale onto the seat where I was sitting and filled the cockpit of the boat.  But we didn't capsize and Lisa volunteered to find the bailing bucket and undo her work.  I found out that my boat sails pretty well full of water.  Jeff's boat has a self bailer that sucks the water out when you are moving but mine does not.   Lisa got most of the water out and we headed back with a fun story to tell.
\\
\\
We hope to have another sailing adventure in August.  I'll post it on the listserve in case any of you want to join us.
\\
\\
-- Captain Bill Baxter

\hrule

\subsect{ECP Sailing School - Mike Ciccone}
It was a pleasure to take part in the first ECP sailing School.  The class met at 7PM on Tuesday June 11 2013 at Bill Baxter's House in Ross Twp.  Four students arrived for the class session.  The class began on the deck when students received a Boy Scouts of America Handbook on Sailing Small Boats.  A few questions were asked and then we went into the back yard where Bill and Jeff Baxter's Sailboats are kept.  The 6 people split into groups of 3 and we began by uncovering the boats.  Bill's Boat is a bit larger and has room for 4 people, while Jeff has a smaller but faster boat with room for 2 people. 
\\
\\
Students were directed through the activities needed to prepare the sailboat for launch.  Masts were set and rigging was attached.  At least one ECP member was a bit disappointed at the absence of a boomvang on either boat.  We were informed that ropes are not called ropes- but instead called lines in sailing.  There are Halliards to raise sails, and sheets to change the tension on sails when underway.  One of the first things we did was open the sails.  On Bill's Boat the sail was neatly folded in a bag (about the size of a tent bag).  On Jeff's Boat the sail furled around the mast, which neatly stored in his boat (fitting into 2 built in holes.  A simile pull on a line was enough to unfurl Jeff's Sail.  Bill's Sail needed to be rolled out and attached to the mast.  At this point it could be raised with the Halliards.  It is interesting to note that rigging for masts attaches to the gunwales of the boats via a small clip that looks like a keychain which fits neatly into a grommet.  Each boat has 2 sails (main sail and Jib Sail).  The sails receive reinforcement with a device called a batten (just a piece of wood or plastic that fits into a slot on the sail. 
\\
\\
Once the boats were fully rigged up with masts rising (? 15-20' into the air) we practiced attaching them to hitches at the back of Bill and Jeff's Vehicles.  We discussed sailing terminology- which is quite extensive.  The back is the stern, the front is the bow.  The 2 sides are called starboard and port.  Each boat was equipped with a centerboard- a small rudder that lowers from a horizontal to vertical position from a slot in the middle of the sailboat.  The centerboard is needed to add stability and help with steering the boat.  other types of sailboats can have dagger boards (small removable boards) that are placed in a hole and go straight down into the water, or keels.  Keels are used mostly in larger boats where a large amount of weight is needed below the sailboat to stabilize it.  We learned that there is a "no go zone" of 45 degrees (where the sail boat will not move because no wind is reaching the sails.  Another area where the sailboat will lot move much is when the boat is facing directly into the wind- called irons.  At this point the sails flap back and fourth rapidly- called luffing.  The fastest speeds are on a broad reach.
\\
\\
Time went fast, and before we knew it- darkness was setting in.  At this point we unhitched and derigged the boats to prepare them for storage until our 6/15/2013 voyage at Moraine State Park.  We then went into Bill's house for beverages and to discuss the logistics for Saturday's Trip.  Around 9:35PM the group began to break up to go home.
 
\begin{Figure}
 \centering
 \includegraphics[width=0.95\linewidth]{pics/mike1.jpg}
 \captionof{figure}{ECP Sailing boat}
\end{Figure}
 
Four days later on Saturday 6/15/2013 three students arrived at Bill Baxter's house at 9AM to go sailing.  Two of the students were not there, however a new student was present.  Around 10AM or so the boats were hitched and we were off to Moraine State Park (South Shore).  Upon arrival- sometime around 11AM- we rigged up the sailboats for sailing in the parking lot, and hauled them to the lake.  We were greeted with a nice variable wind (up to about 12MPH) a bit stronger than the 7MPH predicted by the weather forcasters.  This was kind of unusual since the wind usually picks up in the afternoon at Lake Arthur.  The wind at Lake Arthur is also extremely variable changing direction all over the compass every few minutes. 
\\
\\
Very shortly after getting on the lake, the wind filled the sails and we began moving pretty fast.  Wind only lasted a few minutes at most, before we would slow down and wait for the next burst of wind.  You could see the wind coming on the lake by the ripple patterns.  We went across the lake once or twice. Getting back required tacking or sailing back and fourth into the wind (but not directly into the wind).  The morning was cool (in the 60s).  After lunch, however the wind died off completely, and it warmed up to about 80 degrees.  This made for a slow trip, but also made it easier for students to switch between Jeff and Bill's boats when they were together in a flotilla.  
\\
\\
After several hours of little or no wind and watching some larger boats with spinnikers (big parachute like sails at the front of the boat) move a little, we decided to head in.  This required that we paddle (yes there are paddles in the sailboats) to shore once we got close.  Around 3PM or 3:30PM (I forget exactly when) we docked the boats and loaded them onto the trailors of Jeff and Bill's vehicles.  We then has some iced tea, and drove back.  Considering the less than optimal wind conditions this class went over very well with students learning a lot of terminology, and all students having the opportunity to steer and sail both boats.  Thanks Bill and Jeff for a great class.
\\
\\
--Mike Ciccone
\hrule

\subsect{ECP Sailing School - Ron Edwards}
Over the past couple of years, Bill Baxter has poked around for possible interest in a sailing school.  
On June 6 I got my acceptance letter into the school.  Yar! Santosh Chandrasekaran, Bill Brose, Mike Ciccone, Lisa Falenski and I were the official students.  Ahoy mateys!  (Too much, too soon?)
\\
\\
Bill provided an electronic copy of our book along with reading assignments in advance so that we would be prepared for class.  On Tuesday June 11 we gathered at Bill Baxter's house for our hands-on training.  After reviewing the readings and brief Q\&A, we started assembling Bill's boat, followed by Jeff's.  Both are similar craft (sloops) with enough differences to get a sense of variety in construction, parts, and purpose.  There are a LOT of terms in sailing (possibly even more than climbing!) and Bill and Jeff put us at ease covering the essential ones but assure we didn't need to know every one (such as the individual names for each edge and corner of a main sail).  My favorite term is the "boom vang" and I requested everyone call me that for the rest of the class.
\begin{Figure}
 \centering
 \includegraphics[width=0.95\linewidth]{pics/ron1.jpg}
 \captionof{figure}{Bill Baxter's House}
\end{Figure}
After some hypothetical situations of wind direction and turning and tacking we disassembled the boats and finished up the lecture over some beers (my kind of sport!). On the following Saturday, June 15, the weather was clear and sunny, though little and low winds were predicted.  Lisa and Bill were not able to attend, but Santosh, Mike, and I were ready to set sail.  Apparently sailing is not an early bird type of sport.  Winds are best mid-day, so it's usually a casual start, casting off around 10am.  I wouldn't need a headlamp or even an alarm to be ready for this outing!
\\
\\
We headed to Lake Arthur, part of Moraine State Park, just a short hour drive north of Pittsburgh. There is also nice technical (rocky) mountain biking, trail running, and McConnell's Mill is nearby for rock climbing.  The 3,225-acre Lake Arthur has 10 public boat launches. Sailing conditions are often ideal, and races, regattas and sailing instruction classes are held throughout the summer.  There is a sailing club at the lake with boats for rent.  Sail boats share the water with motor boats (up to 20HP).
\begin{Figure}
 \centering
 \includegraphics[width=0.95\linewidth]{pics/ron2.jpg}
 \captionof{figure}{Getting Ready to launch}
\end{Figure}
We pulled into the launch area and quickly got to work assembling the boats as we had learned previously.  Backing the boats into the water we applied our hitch skills to secure them to the cleats of the dock.  We dropped the center boards, donned PFDs, boarded, hoisted the jib and cast off!
\\
\\
The wind was higher than predicted, we guessed between 5 and 13 MPG, and soon under mainsail, we were tacking across the lake.  We soon learned that the lake's wind can be fickle, pocketed with steadier or calmer winds and being more reliable near the center.  After the basics had been covered and many turns made, we students were given the responsibility of manning the tiller (rudder control).   Amazingly, no other boaters were harmed during our shifts.  We also managed to come alongside each other's boats twice to jump ships and allow each to experience the other boat.
\\
\\
By noon the wind had died down significantly and we were drifting more than harnessing wind in our sails.  Lunches came out, sun beat down, and even a nap was had.  Floating in the middle of a calm lake is about as relaxing as it gets.  Eventually we decided to call it a day and used oars to paddle our way back to shore.  Beers aren't permitted in the state park so we took a short hike into the woods, accidently uncapped some and emptied them appropriately before packing the bottles out.
\\
\\
There is talk of heading up to Lake Erie in the late summer to sail the bay.  Many thanks to Bill and Jeff for graciously sharing their knowledge, boats, and time to introduce us new sailors to a wonderful activity.
A bit of wisdom, appropriately cast: "A Smooth Sea Never Made a Skillful Sailor".
\\
\\
-- Ron Edwards

\begin{Figure}
 \centering
 \includegraphics[width=0.95\linewidth]{pics/ron4.jpg}
 \captionof{figure}{Ahoy! Ron and Jeff}
\end{Figure}

\hrule

\clearpage
\pagebreak

\sect{Club Business}
This sections contains information on club business and other official matters. 

\subsect{Meeting Schedule and Locations}

\Large
\textbf{JULY GENERAL MEETING}\\
Thursday, July 11, 7:30PM\\
Frick Park Pavillion
Forbes and Braddock Ave
\\
\\
\textbf{JULY BOG MEETING}\\
Thursday, July 18, 7:30PM\\
Rush Howe's House

\normalsize


\subsect{JULY GENERAL MEETING AGENDA}
\textbf{OFFICER REPORTS}\\
\textit{\textbf{President, Rush Howe}} : No Report. See June General Meeting Minutes \\
\textit{\textbf{Vice President, Bill Baxter}}: No Report. See June General Meeting Minutes \\
\textit{\textbf{Secretary, Phil Sidel}} :  See June General Meeting Minutes \\
\textit{\textbf{Activity Chair, Ron Edwards}} : No Report. See June General meeting minutes. \\
\textit{\textbf{Treasurer, Kathleen}} : Report Submitted. To be included in next newsletter. \\
\textit{\textbf{Equipment Chair, Derek Stuart}} : No Report \\
\textit{\textbf{Editor, Subhodeep Moitra}} : Whazzap!!!
\\
\\
\textbf{APPOINTEE REPORTS}\\\\
\textit{\textbf{Membership Coordinator, Martha Gray}} : The following have applied for Membership:

\begin{center}
	\begin{tabular}{c}
		Mike Whaley
	\end{tabular}
\end{center}

\textit{\textbf{Advertising, Tara Powers}} :  No Report \\
\textit{\textbf{Environmental, Ginette Vinski}} : No Report \\
\textit{\textbf{Historian, Phil Sidel}} : No Report \\
\textit{\textbf{Librarian, Phil Bradenbach}} : See Librarian section\\
\textit{\textbf{Webmaster, Tom George}} : No Report\\
\\
\hrule


\subsect{JUNE GENERAL MEETING MINUTES} 
ECP GENERAL MEETING\\
JUNE 13th 2013\\
David Lawrence Shelter, Schenley Park
Meeting Opened 8:15pm \\
33 current \& former members \\
3 guests in attendance\\
Derek Stuart taking notes for minutes.
\\
\\
\textbf{OFFICER and APPOINTEE REPORTS}\\
\\
\textit{\textbf{President Rush Howe}}: "We are here"
\\
\\
\textit{\textbf{  VP Bill Baxter}}, - 
\begin{itemize}
\item Tonight, Slide Show on Backcountry Skiing in Tetons
\item No program for July Meeting
\item No electricity  at Frick Park Forbes \& Braddock 	
\item August meeting will be in North Park.
\end{itemize}


\textit{\textbf{Treasurer Kathleen Prigg:}}    Reported basic account totals and handed in a  detailed  report as of May 1, 2013

\begin{center}
	\begin{tabular}{l l}
	Bogel Fund	&	\$10,100.03 \\
	MBEG Fund	&     \$2,276.55\\
	Equipment Fund	  &   \$1,445.84\\
	General Fund	  &   \$7,541.90\\
		TOTAL &	\$21,364.32
	\end{tabular}
\end{center}
   (Note - This report differs from the report published in June issue of The Explorer, 
    though both are dated May 1.)   MBEG fundraier was great!  \$955 taken home plus money still to come 
   from auction and contributions.  Rush thanked everyone involved in the event.  Everyone had lots of fun.
\\
\\
\textit{\textbf{Secretary Phil Sidel:}}  Was ready to read BOG Minutes (per Constitution) but membership voted to accept minutes as published in the June issue of The Explorer.

\textit{\textbf{Activities Chairperson, Ron Edwards:}}
	\begin{itemize}
\item   Fun activities prior to tonight's meeting
\item   Women's Trad Weekend at Seneca - Late June
\item   Pgh SCUBA divers - extending invitations - Summersville - posted on listserve
\item    MS150 - July 20-21 - Maggie's Marauders will be doing Keystone Country Ride
	Riders are invited to join the team - others are invited to donate (each team 
	member commits to raise \$300 - John Zolko was selling popcorn as fundraiser).So far there are over 60 riders signed up for the team. 	Training Rides every Wednesday 5pm, North Park and Sundays at varying locations.  
\item   Flotilla - August 3-4, Contact Tony Vinski
	\end{itemize}


\textbf{APPOINTEE REPORTS}
\\
\\
 \textit{  \textbf{Historian - Phil Sidel}} - Nothing to report
   
\textbf{OLD BUSINESS}
\begin{itemize}
\item    John Zolko reported -	More work to be done by climbers on church roof in East Liberty. 	Nice coverage in Post Gazette. 
\item    Ron Edwards reported - 	New Logo project is still open - submissions will be reviewed by BOG
\end{itemize}
 
\textbf{NEW BUSINESS}
\begin{itemize}
\item    Tom Prigg proposed we take older documents and produce a digital archives.  	Probably will require professional archiving services.    	Must  identify tasks and get cost quotations. General approval subject to cost.  
\end{itemize}

Meeting Adjourned - 8:45pm\\
\\
\textbf{PROGRAM:}\\
    Kevin Chartier presented a wonderful slide show of Ski Mountaineering in the Tetons, February,2013.  

\end{multicols}

\hrule 


\clearpage
\pagebreak

%\subsect{Treasurer Report}
%
%\begin{Figure}
% \centering
% \includegraphics[width=\linewidth]{pics/treasure1.pdf}
%\end{Figure}
%
%
%\begin{Figure}
% \centering
% \includegraphics[width=\linewidth]{pics/treasure2.pdf}
%\end{Figure}

\pagebreak

\sect{Other Announcements}

\subsect{Adventures and Scientists for Conservation}
I would like to bring your attention to the following group, \textbf{"Adventurers and Scientists for Conservation"}.  The organization describes itself as "\textit{Dedicated to improving the accessibility of scientific knowledge through partnerships between outdoor enthusiasts and scientists.}" They have an interesting model for synergizing the two communities - \textit{"We ask adventure athletes who are travelling around the world to collect data as they explore the outdoors.  The data we collect goes to researchers who have asked for our help in getting expensive, time consuming, and difficult to reach information from the remote corners of the globe. "}
\\
\\
You can find more details on their website at \url{http://www.adventureandscience.org/}.
\\
\\
Phil Sidel(ECP Secretary) had this to say with regard to the ECP's relation to science - "\textit{...The ASC organization seems to have goals and program that fits perfectly into the "Purpose" of ECP as defined in our constitution..."}:
\\
\\
Here is an excerpt from the ECP constitution : 
\\
\\
The ECP was organized in 1947 to promote exploratory science and adventure. The functions of the club shall be:
\begin{enumerate}
\item To promote explorations at home and abroad.
\item To educate interested persons in exploratory science.
\end{enumerate}

This seems like a wonderful opportunity! A chance to contribute to the body of scientific knowledge the next time we Explorers go an adventure.
\\
\\
-- Subhodeep Moitra

\hrule

\subsect{Paddling Trip - Mike Ciccone}
Here is a repeat of a popular paddling trip I first led on 7/31/2011.  It is being posted to the Butler Outdoor Club, WPPSA, Outkasts, and the Explorers Club and may also be posted on the Kay yackers Meetup group depending on response.
 \\
 \\
\textbf{Leader}: Mike Ciccone \\
July 27, 2013 Saturday 10AM
\\
 \\
Allegheny River/ Mahoning Creek Kayak/ Canoe Trip (up to 21 miles; 8.5 upstream \& 8.5 back with 2 upstream and 2 back on Mahoning Creek)
 \\
 \\
Paddle upriver on the slack water of the Allegheny River for 2 miles paralleling the Armstrong BikeTrail. Turn up Mahoning Creek, easy slack water, and wind your way upstream past camps into sparkling clear water in this deep Gorge. Around 2 miles is where the creek becomes rocky. We will stop for a break/ lunch on a rocky island.  If the water is high enough we can run the rapids (easy class I or II).  At this point we will paddle back down Mahoning Creek re-entering the Allegheny River.  Continue paddling North up to 7 miles past numerous small towns, and travel through a very scenic stretch of river with large hills on both sides. When in sight of the Lock 9 Dam (the largest and Northernmost Lock on the Allegheny River) we will paddle back downstream on the opposite side of the river back to our starting point.  Please contact Mike Ciccone at 724-584-8045 if interested in this trip, and to obtain directions to theTempleton, PA Fish and Boat Commission Launch (put in/ take out) if you are not familiar with this location or try.
 \\
 \\
\url{http://www.armstrongtrail.org/pdf/parking/templeton.pdf}
 \\
 \\
There is no shuttle on this trip and estimated trip time is 5.5 to 6.5 hours including stops for lunch or if people just want to stop.  At the time of publication it is unknown whether or not boats will be available for rent this year.  I can take your name as tentative if you need to rent a boat, and get back to you as soon as I know about ability.

\hrule
\pagebreak

\subsect{North Country Trail Meet}

\begin{Figure}
 \centering
 \includegraphics[width=\linewidth]{pics/extrav.pdf}
\end{Figure}

\pagebreak
\clearpage
\subsect{Librarian Section - Phil Bradenbach}
New Books in the Club Library\\
July 2013
\\
\\
	Five new books have been added to the clubs book collection, a couple about technical scuba diving, two covering hikes on the Appalachian Trail and one dealing with the aging outdoors person. 	Hopefully, there will be something in this selection that will excite you.  The books are here for you to read, make use of them!
\\
\\
	The Last Dive, by Bernie Chowdhury tells the story of a father and son team of scuba divers who are attempting to uncover the secrets of a mysterious sunken U-boat off the coast of New York.  The boat has been laying on the ocean bottom since WWII, they are hoping to achieve recognition in uncovering the details as to how it got there, when it got there and why it was sunk.  Unfortunately, they paid the ultimate price in their search.   
	
\begin{Figure}
 \centering
 \includegraphics[width=0.9\linewidth]{pics/books.png}
\end{Figure}
	
	
	This book reads like a fictional adventure story, but it's not.  It pulls you into the exciting world of deep-sea diving.  I found it to be a very enjoyable read and hopefully you will also!
	Staying in the same field, technical diving, The Six Skills and Other Discussions by Steve Lewis is a book that covers creative solutions for scuba diving.  Steve lays out the fundamental principles and skill sets needed in modern  technical diving.  The book reads like a series of educational, entertaining, information rich essays.  These are rewarding to both novice and experienced divers alike.
	Donna Billings book, Red and Purple Hiking Boots( An Older woman's Trek to "It's Never Too Late"), tells about some of her exploits and adventures and how she overcame fear and the worry of doing things in ways that weren't conventional.  As you read this, she supplies questions you might want to ask yourself and help explain why you shouldn't stop going into the wild, even though your years are adding up!
\\
\\
	The next couple books I really enjoyed.  They are books dealing with hiking the Appalachian Trail.  For some reason, I've always enjoyed reading books that tell about hikers journeys from Georgia to Maine on the AT.  Perhaps it is because I've hiked sections of the trail myself, perhaps it is because I've always thought it would be a great experience to do it myself, perhaps, it is just for the inspiration I get from these books.   Anyway, I almost always have enjoyed books dealing with these hikes.  Here are 2 more, hopefully you will enjoy them also!
\\
\\	
	Walking Home, by Lucy and Susan Letcher (a.k.a. Isis and Jackrabbit) is the second book by these sisters.  It is a sequel to Southbound, the story of their hike from Maine. (This book is also in the club library!) In the first book, the sisters decide to do this hike sans shoes, earning themselves the name and a certain bit of  trail fame, as the" Barefoot Sisters".   After arriving in Georgia, they decide instead of buying a cheap car to drive back home as they originally planned, they would just hike back.  After doing the trail one way, the return should be easy!  "fraid not" Their return hike is what the second book covers.  In this hike, they end up coming against a whole bunch of new difficulties and problems.  The seasons are different, new wildlife is encountered and a whole bunch of new hikers to meet and befriend.  
\\
\\
	You might not want to hike the hike the same way as them, but perhaps this book might push you towards doing a section or two of the trail, if not all of it!  You don't have to do it without shoes!
\\
\\
	Our final book this month is my favorite of the five.  In Beauty May She Walk,(Hiking The Appalachian Trail at 60) by Leslie Mass tells the story of a woman who decides to finally do the Appalachian Trail, a desire she has had for decades.  She talks with her father who had instilled the joys of the outdoors in her, shortly before he died.  She told him she was finally going to do the trail and afterwards, decides to make her hike a memorial to him.  Having just passed her 60th birthday, she realizes that it wasn't going to be easy, but it was still doable!  AS she starts out, she learns a lot by experience, the problems of most new hikers on the trail, over packing, learning about protecting your food from the wildlife on the trail and learning to trust you instincts.   During her hike she meet a great bunch of fellow hikers  and tells us about them.  She also tries to include her friends and family in her hike.  She finds out that including these "non-hikers" tends to mess up her own hiking style.  She finds out that you have to come to a certain understanding on how to "hike your own hike".
\\
\\
	After 9/11 she finds out that things have changed, even on the trail and has to take this into account on how she does her hike.  The book is a great story, it is definitely inspirational, if this woman can do it in her 60's, all you "kids" in your 20's should have no problem at all.  Go on, go out and "Hike Your Own Hike!"  Read the book hope ya like it, I sure did!
\\
\\
	As is always the case, if you see a book that interests you, give me a call or drop me an e-mail and I'll get it to you.  All that is required is a club membership and a small deposit.  You will get the deposit back when you return the book.  If you have any ideas for new books, (I am always open to new ideas!) drop me a line and I'll look into getting them. 
\\
\\
Happy reading,\\
Phil B. ECP Librarian\\
(412)486-1450\\
\url{Booksandmaps@Yahoo.com}

\hrule

\clearpage
\pagebreak


\appendix

\section{Contact Information}

\subsection{Activities Coordinators}
For some of our activities, we currently have no active coordinators. If you think you might be able to coordinate
and lead these activities, contact Activities Chairperson Ron Edwards ecp@edwardsjr.com (412-327-2084)


\begin{center}
    \begin{tabular}{ | l | l | l | }
    \hline
    \textbf{Activity} & \textbf{Contact} & \textbf{Email} \\\hline
	Backpacking & Suraj Joseph & surajj1234@gmail.com \\ \hline
	Biking - Mountain & Rush Howe & rushhowe@yahoo.com \\ \hline
	Biking - Road & TBD & TBD \\ \hline
    Caving  & Doug Fulton & fulton12b@yahoo.com\\ \hline
    Fly Fishing & Bruce Cox & brcox33@comcast.net\\ \hline
    Ice Climbing & Tom Prigg & Tom.prigg@gmail.com\\ \hline
    In-Line Skating & Robin Kamin & ekamin@verizon.net\\ \hline
    Mountaineering & Sam Taggart & samuel.taggart@gmail.com\\ \hline
    Paddling - Flat Water & Nick Ross & Lnickross@gmail.com\\ \hline
    Paddling - White Water & Barry Adams & bj2adams@juno.com\\ \hline
    Rock Climbing & Ron Edwards & ecp@edwardsjr.com\\ \hline
    Rowing & Bob Dezort & bobdezort@verizon.net\\ \hline
    Trail Running & Brian Ottinger &  brian-ecp@ottinger10.com\\ \hline
    Sailing & Bill Baxter & billybax@yahoo.com\\ \hline
		   &  Jeff Baxter& jeffreywbaxter@yahoo.com\\ \hline
	Skiing & Downhill  &  Lindsay Hastings LMH239@gmail.com\\ \hline
	Yoga  &  Allison Pochapin & a.pochapin@gmail.com\\ \hline
            &  Elise Nolan & elise.nolan@gmail.com\\ \hline
	Adventure Racing & & \\ \hline
	Skiing  Cross-Country & & \\ \hline
	Rafting & & \\ \hline
	SCUBA Diving & & \\ \hline
	Sea Kayaking & & \\ \hline
	Triathlon Training & & \\ \hline
	Orienteering (proposed) & & \\ \hline
	Geocaching (proposed) & & \\ \hline
	Skydiving (proposed) & & \\ \hline
	 \hline
    \end{tabular}
\end{center}

\paragraph{What does an Activity Coordinator do?}
General advocacy and point of contact for the activity by helping current, new, or prospective members
get connected with others in the club also interested in the activity and any events planned for it.
\begin{enumerate}
\item Seek out new members at general meetings (during the break or at the after-meeting) that
indicate the activity as one of their interests and introduce yourself as that activities
"coordinator" that can help them connect with and explore the activity.
\item Answer incoming emails to the club (sometimes redirected from officers or others who receive
the inquiry) related to the activity. This is rare.
\item Spread the word if you see any local events or news related to the activity that you think
members would like to know by posting (or reposting) on the Listserv and mentioning at any
General Meeting that you attend. In some cases, create an Event (takes 2 minutes) on the ECP
calendar to give that event extra visibility.
\item Encourage any members that are planning an event on their own or as a small/closed group to
consider opening it up to the general membership in the form of an official Event (and have
them create the Event)
\item If possible, directly or indirectly ensure that at least one Event for your activity occurs every year.
\end{enumerate}

\fbox{\parbox{\linewidth}{
{\Large\bf ECPers!!}\\
Contact the Coordinator for activities in which you want to participate.\\
There are many activities which are not listed in the calendar,
because the arrangements are spur-of-the-moment, or because\\
they are regular, periodic (e.g., weekly) activities.\\
\\
Some examples are:\\
Weekly In-Line Skating Sessions\\
(Contact Coordinator Robin Kamin)\\
Yoga Sessions in Frick Park\\
(Contact Yoga Coordinator Elise Nolan)\\
Flat Water Canoeing trips\\
(Contact Coordinator Nick Ross)\\
Mountain Biking Runs\\
(Contact Coordinator Rush Howe)\\
And many others\dots}}

\subsection{Officers}
\begin{center}
    \begin{tabular}{ | l | l | l | l | }
    \hline
    \textbf{Position} & \textbf{Name} & \textbf{Email} & \textbf{Phone} \\\hline
	President & Rush Howe & president@pittecp.org & 412-983-5265 \\\hline
	Vice-President & Bill Baxter & vicepresident@pittecp.org & 412-926-8261 \\\hline
	Secretary & Phil Sidel & sidel.climbing@gmail.com & 412-521-9570 \\\hline
	Treasurer & Kathleen Prigg &  kathleen.prigg@gmail.com &  724-584-4013 \\\hline
	Activities Chair & Ron Edwards & activities@pittecp.org & 412-327-2084 \\\hline
	Equipment Chair & Derek Stuart & equipmentchair@pittecp.org & 585-727-1258 \\\hline
	Editor & Subhodeep Moitra & subho@cmu.edu & 412-721-5305 \\\hline
    \end{tabular}
\end{center}

\subsection{Appointees}
The ECP Appointees are persons appointed by the president to fill key positions in the club. In
addition there are appointed Activity Coordinators and Special Committees.

\begin{center}
    \begin{tabular}{ | l | l | l | l | }
    \hline
    \textbf{Position} & \textbf{Name} & \textbf{Email} & \textbf{Phone} \\\hline
	Advertising & Tara Powers &	tarasmagicalpowers@gmail.com &  \\\hline
	Environmental & Ginette Vinski & ginette@vinski.net & 412-366-4925 \\\hline
	Historian & Phil Sidel & sidel.climbing@gmail.com & 	412-521-9570 \\\hline
	Librarian & Phil Breidenbach & Booksandmaps@yahoo.com & 412-486-1450 \\\hline
	Membership & Martha Gray & 	graymf@gmail.com &  \\\hline
	Webmaster & Tom George & webmaster@pittecp.org & 412-831-4711 \\\hline
	\end{tabular}
\end{center}



\pagebreak
\clearpage


%----
\end{document} 

